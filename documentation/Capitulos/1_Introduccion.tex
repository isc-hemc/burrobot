\chapter{Introducción}

\section{Presentación}

    En 1950, Alan Turing publica un artículo de Investigación llamado: Computing Machinery and Intelligence [1]; en la investigación, se cuestiona si las computadoras podrían pensar, a aprender e interactuar con el usuario y además define la inteligencia de computadoras como: “una computadora humana capaz de procesar instrucciones y tener conciencia”. Debido a la investigación de Turing, se inicia las bases científicas para realizar las primeras investigaciones sobre inteligencia artificial y la comunicación entre humano computadora de una forma natural. Turing predijo que al final del siglo, el uso de las palabras y las opiniones educadas podrían alterarse tanto que se iba lograr hablar con las computadoras, estas máquinas lograrían pensar sin esperarse una contradicción [2].
    
    El Procesamiento de Lenguaje Natural, o también conocido como computación lingüística, le permite a la computadora interpretar el lenguaje humano basado en el razonamiento, aprendizaje y entendimiento. El NLP, fue transformado por investigadores para poder construir un modelo exitoso en la traducción humano-computadora con lenguajes empíricos de datos. Para entender el lenguaje humano se han desarrollado tres técnicas:
    \begin{description}
        
        \item[Machine Translation:]Es la forma como la computadora traduce lenguajes humanos y logra descomponerlos en una estructura semántica entendida por una computadora. Según Hirschberg, esta tecnología ha avanzado gracias al Deep Learning, que consiste en entrenar un modelo con diferentes representaciones para optimizar un objeto final en este caso, las traducciones [3], además, permite el uso de tecnologías actuales como: Google Translate y Skype Translator.
        \item[Speech Recognition:] Es el proceso de convertir una señal del diálogo en una secuencia de palabras, esto por medio de un algoritmo implementado por un programa de computadora [2]. La tecnología de speech recognition, ha hecho posible a la computadora responder por comandos de voz y entender el lenguaje natural como lo hacen los asistentes virtuales de los teléfonos y los parlantes como Alexa. Existen tres formas del speech recognition: palabras insoladas, palabras conectadas, y diálogo espontáneo [2].
        
        \item[Speech Synthesis:] Es la forma en que la computadora pasa de texto a diálogo, y el software debe comprender la entonación, pronunciamiento y duración del diálogo.
    
    \end{description}
    
    Esto nos lleva a la pregunta ¿Qué es y para qué sirve un chatbot?
    El primer Bot fue creado en 1996 por Joseph Weizenbaum, experto en informática y de nacionalidad Germano-Americana. Eliza era el nombre de este Bot y fue diseñado para comportarse como un terapeuta, claro, sin ningún tipo de inteligencia artificial y con respuestas algo limitadas, pero sin importar esto, ya era capaz de hacer preguntas como “¿Y eso cómo te hace sentir?”. Un chatbot es un servicio o herramienta con el que puedes comunicarte mediante mensajes de texto. Utiliza las bases de la Inteligencia Artificial y el Procesamiento de Lenguaje Natural haciéndolo capaz de entender lo que estás intentando decir y responde con un mensaje coherente, relevante y directo relacionado con la tarea o petición que le estás solicitando gracias a los avances que se han tenido desde la década de los 50’s.  Lo que los hace tan relevantes hoy en día es que: 
    
    Los medios que utilizan para llevar la conversación suelen ser aplicaciones de mensajería instantánea como Facebook Messenger y WhatsApp, que están posicionados como principales medios de comunicación debido a su popularidad. Además, en los últimos años estos medios se han potenciado debido a la posibilidad de comunicar directamente a las personas con instituciones y así establecer una atención personalizada.
    
    Los avances en la inteligencia artificial, el aprendizaje automático y el procesamiento del lenguaje natural,  permite a los bots dar seguimiento a una conversación.
    
    Los chatbots modernos no dependen únicamente de los mensajes de texto. También pueden enviar imágenes, enlaces, etc., proporcionando una experiencia similar a una aplicación.
    
    Esto les permite ser utilizados para diversos propósitos, tales como compras, servicio al cliente, noticias, juegos, campañas de mercadotecnia, entre otras.

\section{Motivación}
    Cada día se realizan una gran cantidad de preguntas relacionadas con temas administrativos por parte de los alumnos del Instituto Politécnico Nacional, los cuales buscan soluciones rápidas y precisas. Actualmente el instituto asigna recursos humanos y técnicos, destinados a dar soporte y solución a las dudas de los alumnos, algunas formas que se tienen para atender a los alumnos son: presencial, asistencia telefónica, redes sociales (mensajería instantánea y grupos), correo electrónico y páginas web. Sin embargo, el uso de recursos humanos para atender esta problemática demanda una gran cantidad de tiempo para dar solución a cada una de las consultas, por lo que no es posible dar una respuesta en tiempo y forma a muchas de las peticiones por parte de los usuarios.
    
    La propuesta de solución consiste en la creación de un prototipo de sistema de chatbots que lleva por nombre Burrobot. Este sistema será capaz de sostener conversaciones con usuarios a través de la plataforma de Facebook, para brindarles respuestas automáticas a sus consultas más recurrentes de manera personalizada y con un tiempo de respuesta inmediato y garantizando una alta disponibilidad.
    \section{Planteamiento del problema}
    
    Muchos de los procesos para realizar trámites del ámbito de la gestión educativa son desconocidos por parte de los alumnos del Instituto Politécnico Nacional, naturalmente ellos recuren a sus propios recursos para poder obtener la información respectiva, sin embargo al no haber un medio de consulta enfocado exclusivamente a la resolución de estos problemas donde toda esta información se encuentre centralizada y las respuestas de este medio proporcione información pronta y de calidad. esto puede implicar que la información obtenida por los alumnos no sea: completa, oportuna, relevante, accesible, entendible, clara y precisa. esto puede derivar en una serie de problemas que van desde pequeños contratiempos hasta no poder realizar el tramite.
    
    Por otro lado el personal administrativo del área de gestión escolar también se ve afectado, ya que ellos son los que prestan este servicio, lo cual es un problema en fechas criticas en la cual la demanda del área aumenta a tal punto que no es posible poder atender a todas las consultas. y en el caso de que el alumno no reúna los requisitos necesarios debido su falta de información o documentación 
    \section{Objetivos}
    \subsection{Objetivo general}
    Desarrollar un prototipo de agente conversacional que interactúa con los usuarios utilizando técnicas de Inteligencia Artificial y Procesamiento de Lenguaje Natural integrado a la  de mensajería Messenger.

    \subsection{Objetivo específico}

    \begin{itemize}
        \item Identificar las necesidades de gestión escolar que puedan ser resueltas con la aplicación.
        \item Procesar recursos lingüísticos para el entrenamiento del chatbot.
        \item Construir el modelo de lenguaje.
        \item Integración con el servicio de mensajería Messenger.
    \end{itemize}

\section{Estado del arte}
    \subsection{Investigación}
        
        \begin{longtable}[c]{| >{\centering\arraybackslash}m{2cm} | >{\centering\arraybackslash}m{6cm} | >{\centering\arraybackslash}m{4cm} | >{\centering\arraybackslash}m{3cm} |}
        
            \caption{Estado del arte - Investigaciones.\label{long}}\\
            
            \hline
            {\bf Sistema} & {\bf Descripción} & {\bf Características} & {\bf Estado actual}  \\ \hline
            \endfirsthead
            
            \hline
            \multicolumn{4}{| c |}{Continuación de la tabla: \ref{long}}\\ \hline
            {\bf Sistema} & {\bf Descripción} & {\bf Características} & {\bf Estado actual}  \\ \hline
            \endhead
            
            \hline
            \endfoot
            
            \hline
            \multicolumn{4}{| c |}{Final de la tabla: \ref{long}}\\
            \hline
            \endlastfoot
            
            {\bf Chatbot for university related FAQs} &
            Chatbot que proporciona una respuesta eficiente
            y precisa para cualquier consulta basada
            en el conjunto de datos de preguntas
            frecuentes utilizando el lenguaje de
            marcado de inteligencia artificial y
            el análisis semántico latente (LSA).
            Las preguntas generales y basadas en
            plantillas como bienvenida / saludos y
            preguntas generales se responderán
            utilizando lenguaje de marcado y otras preguntas basadas
            en servicios utilizan LSA para proporcionar
            respuestas en cualquier momento que sirvan
            a la satisfacción del usuario.& 
            \begin{itemize}[leftmargin=*]
                \item El usuario publica la consulta en chatbot
                \item La coincidencia de patrones se realiza entre la consulta ingresada por el usuario y el conocimiento
            \end{itemize} &
            Proyecto de investigación en desarrollo\\ \hline
            
            {\bf Automated Thai-FAQ Chatbot using RNN-LSTM} &
            
            Se desarrollo un Chatbot de preguntas frecuentes (FAQ) que responde automáticamente a los clientes mediante el uso de una red neuronal recurrente (RNN) en forma de memoria de corto plazo (LSTM) para la clasificación de texto. Los resultados experimentales han demostrado que chatbot podría reconocer el 86.36\% de las preguntas y responder con una precisión del 93.2\%.
            & 
            \begin{itemize}[leftmargin=*]
                \item Impremeditación en pagina web
                \item Redes neuronales recurrentes para procesar las solicitudes de los clientes
                \item Cortos periodos de respuesta
            \end{itemize} &
            Proyecto desarrollado para uso interno de la organización \\ \hline
            
            {\bf Dinus Intelligent Assistance (DINA) Chatbot for University Admission Services} &
            Este documento propone el desarrollo de Chatbot, que actúa como un agente de conversación que puede desempeñar un papel de servicio de estudiante candidato. Este Chatbot se llama Dinus Intelligent Assistance (DINA). DINA utiliza el conocimiento basado como un centro para el enfoque de aprendizaje automático. El patrón extraído puede usarse para proporcionar respuestas al usuario. La fuente de conocimiento está tomada del libro de visitas de Universitas Dian Nuswantoro (UDINUS). Contiene preguntas y respuestas sobre los servicios de admisión de UDINUS.& 
            \begin{itemize}[leftmargin=*]
                \item Entrenamiento rápido
                \item Se basa apartir de una lista de FAQs proporcionadas por la universidad
            \end{itemize} &
            Proyecto de investigación en desarrollo \\ \hline
            {\bf UPIICSA - Trabajo Terminal  2016-07-04} &
            Prototipo de sistema de chatbots capaz de realizar consultas a oraciones para obtener datos que ayuden en la toma de decisiones .
            & 
            \begin{itemize}[leftmargin=*]
                \item Aplicación de escritorio
                \item Análisis de tópicos
                \item Análisis de contexto
            \end{itemize} &
            Trabajo de terminal \\ \hline
            
            {\bf Chatbot BBVA} &

            El chatbot de BBVA es un sistema que conecta con sus clientes, integrado en la app BBVA para ofrecerte una mejor experiencia en la gestión de tus finanzas, resolver tus dudas, y ofrecerte información financiera de forma sencilla. Este chatbot te acerca de una manera distinta toda la información financiera alojada en tu app de banca móvil: mediante una conversación más humana y en un lenguaje cotidiano, sencillo y más cercano.&
            \begin{itemize}[leftmargin=*]
                \item Herramienta integrada a una aplicación móvil
                \item Resuelve preguntas sencillas
                \item Enfocado a la ayuda de gestión en las finanzas de sus clientes
            \end{itemize} &
            Proyecto de investigación en desarrollo \\ \hline
            
            
            {\bf Amazon lex} &

            Amazon Lex es un servicio para crear interfaces de conversación con voz y texto en cualquier aplicación. ofrece las funcionalidades de aprendizaje profundo avanzadas del reconocimiento automático de voz para convertir voz en texto y tecnología de comprensión del lenguaje natural para reconocer la intención del texto &
            \begin{itemize}[leftmargin=*]
                \item Gran escalabilidad sin preocuparse de la infraestructura
                \item Fácil impremeditación en cualquier aplicación
                \item Respuestas y preguntas predefinidas
            \end{itemize} &
            Liberado para uso comercial con un costo de 16 USD por 4000 peticiones de voz y .75 USD por cada 1000 peticiones de texto\\ \hline
            
            {\bf Google dialogflow} &

            Dialogflow sirve para crear interfaces conversacionales para sitios web, aplicaciones móviles, plataformas de mensajería populares y dispositivos IoT. Puede usarlo para crear interfaces (como chatbots y IVR conversacional) que permiten interacciones naturales y ricas entre sus usuarios y su negocio. Los usuarios de Dialogflow Enterprise Edition tienen acceso a Google Cloud Support y un acuerdo de nivel de servicio (SLA) para implementaciones de producción. &
            \begin{itemize}[leftmargin=*]
                \item Gran escalabilidad sin preocuparse de la infraestructura
                \item Perención de sentimientos del usuario
                \item Capacidad de recibir llamadas telefónicas
                \item Fácil impremeditación en cualquier aplicación
                \item Respuestas y preguntas predefinidas
            \end{itemize} &
            Liberado para uso comercial con un costo de 34 USD por 4000 peticiones de voz, 4 USD por cada 1000 peticiones de texto y 6.5 USD por cada 1000 llamadas telefonicas\\ \hline
        
        \end{longtable}

\section{Descripción del documento}

    A lo largo de este trabajo terminal explicaremos a detalle cada uno de los capítulos que conforman este documento. A continuación daremos un resumen de cada uno de estos.

    \begin{description}
        \item[Capitulo 2 Marco conceptual:] En este capitulo explicaremos diversos temas y conceptos técnicos que haremos referencia en los siguientes capítulos, esto para tener una mejor comprensión del tema a tratar de este Trabajo Terminal
        \item[Capitulo 3 Análisis del sistema:] En este capitulo se presenta una estudio de factibilidad, el análisis de requerimientos del sistema, las reglas del negocio, análisis de riesgos y finalmente los s factores tomados en cuenta para la elección de las herramientas de software que emplearemos para el desarrollo del prototipo
        \item[Capitulo 4 Diseño del prototipo:] En este capitulo se presentan los diagramas de procesos, secuencia, casos de uso, clases, entidad-relación. así como el diseño de la infraestructura con el que desplegaremos el servicio. 
    \end{description}



 
 