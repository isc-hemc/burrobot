\chapter{Introducción}

\section{Resumen}
    Los servicios de mensajería han resultado muy útiles para acercar a los clientes de productos o servicios con las empresas, sin embargo, el personal requerido para responder en tiempo real sobrepasa muchas veces las capacidades de la empresa o la institución. La Inteligencia Artificial ha permitido avanzar en la automatización de la comunicación vía mensajería instantánea. Se desarrollará un prototipo de agente conversacional, popularmente conocido como chatbot, que interactúa con los usuarios utilizando técnicas de Inteligencia Artificial y Procesamiento de Lenguaje Natural integrado a la red de mensajería Messenger. La finalidad es apoyar a los alumnos y aspirantes del Instituto Politécnico Nacional orientándolos en las necesidades más frecuentes en el ámbito de la gestión educativa.

\section{Presentación}
    
    Debido a la gran presencia que tienen las redes sociales en nuestro día a día, una gran parte de la población del mundo las han adoptado como un medio tradicional de comunicación, ya que nos ofrecen una forma rápida para poder mantener contacto con familiares, amigos y conocidos.
    
    Los medios que se suelen utilizar para llevar la conversación suelen ser aplicaciones de mensajería instantánea como Facebook Messenger y WhatsApp, los cuales están posicionados como las principales aplicaciones de mensajería debido a su gran y creciente popularidad. Además, en los últimos años estos medios se han potenciado debido a la posibilidad de comunicar directamente a las personas con instituciones y así establecer una atención personalizada.

    Los avances significativos en los campos de la inteligencia artificial, el aprendizaje automático y el procesamiento del lenguaje natural, han provocado importantes cambios de como percibíamos las redes sociales, temas como segmentación del mercado, sistemas de recomendación, análisis de imágenes, agentes conversacionales, entre otros, son hoy en día métodos por los cuales las empresas hacen llegar a las personas sus productos.
    
    
    Uno de los sistemas más usados por los usuarios de Facebook, puesto que han sido de gran ayuda por sus respuestas rápidas, eficientes y concisas, son los agentes conversacionales también conocidos como Chatbots.
    
    Un chatbot es un servicio o herramienta con el que puedes comunicarte mediante mensajes de texto. Utiliza las bases de la Inteligencia Artificial y el Procesamiento de Lenguaje Natural haciéndolo capaz de comprender el contexto de lo el usuario está queriendo decir y responde con un mensaje coherente, relevante y directo relacionado con la tarea o petición que le estás solicitando.
    
    Lo que los hace tan relevantes hoy en d́ıa es que: permite la interacción con usuarios por medio de sistemas de mensajería instantánea. Los chatbots modernos no dependen únicamente de los mensajes de texto. También pueden enviar imágenes, enlaces, vídeos, audios, etc. Esto les permite ser utilizados para diversos propósitos, tales como compras, servicio al cliente, noticias, juegos o campañas de mercadotecnia.

\section{Justificación}

    Cada día se realiza una gran cantidad de preguntas relacionadas con temas administrativos por parte de los alumnos del Instituto Politécnico Nacional, los cuales buscan soluciones rápidas y precisas. Actualmente el instituto asigna recursos humanos y técnicos, destinados a dar soporte y solución a las dudas de los alumnos, algunas formas que se tienen para atenderlos son: presencial, asistencia telefónica, redes sociales, correo electrónico y páginas web. Sin embargo, el uso de recursos humanos para atender esta problemática demanda una gran cantidad de tiempo para dar solución a cada una de las consultas, por lo que no es posible dar una respuesta en tiempo y forma a muchas de las peticiones por parte de los usuarios.
    
    La propuesta de solución consiste en la creación de un prototipo de agente conversacional que lleva por nombre Burrobot. Este sistema será capaz de atender usuarios con técnicas de inteligencia artificial y procesamiento del lenguaje natural a través de la plataforma de Facebook, para brindarles respuestas automáticas a sus consultas más recurrentes de manera personalizada y con un tiempo de respuesta inmediato y garantizando una alta disponibilidad.
    
\section{Planteamiento del problema}
    
    Muchos de los procesos para realizar trámites en el ámbito de la gestión educativa son desconocidos por parte de los alumnos del Instituto Politécnico Nacional, naturalmente ellos recuren a sus propios recursos para poder obtener la información respectiva, sin embargo, al no haber un medio de consulta enfocado exclusivamente a la resolución de estos problemas donde toda esta información se encuentre centralizada, provoca que la información obtenida por los alumnos no sea completa, oportuna, relevante, accesible, entendible, clara o precisa. Esto conlleva en una serie de problemas que van desde pequeños contratiempos hasta no poder realizar el tramite. Las figuras 1.1 y 1.2 son ejemplos de la problemática ya mencionada, donde no hay respuesta alguna por parte de los administradores a las peticiones de los usuarios.
    
    \begin{figure}[H]
       \begin{minipage}{0.5\textwidth}
         \centering
         \includegraphics[width=.7\linewidth]{Latex/Classes/Imagenes/ss-ipn.jpg}
         \caption{Captura de pantalla correspondiente al grupo de Facebook \bf{IPN Zacatenco}.}
       \end{minipage}\hfill
       \begin{minipage}{0.5\textwidth}
         \centering
         \includegraphics[width=.7\linewidth]{Latex/Classes/Imagenes/ss-escom.jpg}
         \caption{Captura de pantalla correspondiente al grupo de Facebook \bf{ESCOM IPN MX}.}
       \end{minipage}
    \end{figure}
    
\section{Objetivos}
    
    \subsection{Objetivo general}
        Desarrollar un prototipo de agente conversacional, el cual se encargará de interactuar con los usuarios de una página de Facebook (la cual lleva por nombre {\bf ESCOM IPN MX}) por medio de su sistema de mensajería instantánea "Messenger", orientándolos en las necesidades más frecuentes en el ámbito de la gestión educativa haciendo uso de técnicas de Inteligencia Artificial y Procesamiento de Lenguaje Natural.

    \subsection{Objetivos específicos}

        \begin{itemize}
            \item Identificar las necesidades de administración escolar que puedan ser resueltas con la aplicación.
            \item Procesar recursos lingüísticos para el entrenamiento del chatbot.
            \item Construir el modelo de lenguaje.
            \item Integración con el servicio de mensajería Messenger.
            \item Garantizar la entrega de mensajes.
            \item El \textit{bot} tendrá como usuarios finales a los aspirantes, alumnos y ex-alumnos de la Escuela Superior de Cómputo \bf{ESCOM}.
        \end{itemize}

\newpage

\section{Estado del arte}

    \subsection{Investigación}

        \begin{longtable}{| >{\centering\arraybackslash}m{3cm} | >{\centering\arraybackslash}m{5cm} | >{\centering\arraybackslash}m{3cm} | >{\centering\arraybackslash}m{2cm} |
        >{\centering\arraybackslash}m{2cm} |}

            \hline
            {\bf Sistema} & {\bf Descripción} & {\bf Características} & {\bf Estado actual} & {\bf Referencia}  \\ \hline
            \endfirsthead
            
            \hline
            \multicolumn{5}{| c |}{Continuación de la tabla: \ref{long}}\\ \hline
            {\bf Sistema} & {\bf Descripción} & {\bf Características} & {\bf Estado actual} & {Referencia}  \\ \hline
            \endhead

            {\bf B. R. Ranoliya, N. Raghuwanshi and S. Singh, "Chatbot for university related FAQs," 2017 International Conference on Advances in Computing, Communications and Informatics (ICACCI), Udupi, 2017, pp. 1525-1530.} &
            Chatbot basado en el conjunto de datos de preguntas frecuentes utilizando el lenguaje de marcado de inteligencia artificial y el análisis semántico latente (LSA)& 
            \begin{itemize}[leftmargin=*]
                \item Resuelve preguntas generales usando un lenguaje de marcado
                \item El modelo de NLP se basa en la obtención de tópicos obtenidos por el LSA
            \end{itemize} &
            Proyecto de investigación en desarrollo - No comerciable &
            \href{https://ieeexplore.ieee.org/document/8126057}{Chatbot for university related FAQs}\\ \hline
            {\bf M. A. Limón, "Construcción de un prototipo de programa personalizado de tipo chatbot en ambiente java", Tesis, UPIICSA, 2016-07-04} &
            Prototipo de sistema de chatbot capaz de realizar consultas a oraciones para obtener datos que ayuden en la toma de decisiones
            & 
            \begin{itemize}[leftmargin=*]
                \item Aplicación de escritorio
                \item Análisis de tópicos
                \item Análisis de contexto
            \end{itemize} &
            Trabajo de terminal - No comerciable &
            \href{https://tesis.ipn.mx/bitstream/handle/123456789/17959/Tesis\%20FINAL.pdf}{Construcción de un prototipo de programa personalizado de tipo chatbot en ambiente java}\\ \hline
            {\bf P. Muangkammuen, N. Intiruk and K. R. Saikaew, "Automated Thai-FAQ Chatbot using RNN-LSTM," 2018 22nd International Computer Science and Engineering Conference (ICSEC), Chiang Mai, Thailand, 2018, pp. 1-4.} &
            
            Se desarrollo un Chatbot de preguntas frecuentes (FAQ) que responde automáticamente a los clientes mediante el uso de una red neuronal recurrente (RNN) en forma de memoria de corto plazo (LSTM) para la clasificación de texto. Los resultados experimentales han demostrado que chatbot podría reconocer el 86.36\% de las preguntas y responder con una precisión del 93.2\%
            & 
            \begin{itemize}[leftmargin=*]
                \item Integración en pagina web
                \item Redes neuronales recurrentes para procesar las solicitudes de los clientes
                \item Cortos periodos de respuesta
            \end{itemize} &
            Proyecto desarrollado para uso interno de la organización - No comerciable &
            \href{https://ieeexplore.ieee.org/document/8712781}{Automated Thai-FAQ Chatbot using RNN-LSTM}\\ \hline
            
            {\bf H. Agus Santoso et al., "Dinus Intelligent Assistance (DINA) Chatbot for University Admission Services," 2018 International Seminar on Application for Technology of Information and Communication, Semarang, 2018, pp. 417-423.} &
            Chatbot que actúa como un agente conversacional que da asesoría a los estudiantes candidatos. DINA utiliza las bases de conocimiento provistas por la institución como parte fundamental a la hora de entrenar sus modelos predictivos. El patrón extraído puede usarse para proporcionar respuestas al usuario. La principal fuente de conocimiento es el libro de visitas de \textit{Universitas Dian Nuswantoro} (UDINUS) que contiene preguntas y respuestas sobre los servicios de admisión & 
            \begin{itemize}[leftmargin=*]
                \item Entrenamiento rápido
                \item Se basa a partir de una lista de FAQs proporcionadas por la universidad
            \end{itemize} &
            Proyecto de investigación en desarrollo - No comerciable &
            \href{https://ieeexplore.ieee.org/document/8549797}{Dinus Intelligent Assistance (DINA) Chatbot for University Admission Services}\\ \hline

            
            {\bf BBVA Smart Assistant Nov 2017.[Blog Digital]. Disponible en : http://www. bbva.com} &
            
            Es un sistema integrado a su producto BBVA móvil. Ofrece a sus clientes una mejor experiencia al usuario al resolver sus dudas sobre gestión financiera, uso de sus aplicaciones y operaciones bancarias. Debido a su conectividad entre las diversas aplicaciones de la empresa, facilita toda la información financiera alojada en los sistemas de banca móvil mediante una conversación más humana, en un lenguaje cotidiano y consultas por medio de comandos de voz &
            \begin{itemize}[leftmargin=*]
                \item Herramienta integrada a una aplicación móvil
                \item Resuelve preguntas sencillas
                \item Integración con redes sociales
                \item Comandos de voz
                \item Enfocado a la ayuda de gestión en las finanzas de sus clientes
            \end{itemize} &
            Sistema de uso interno - No comerciable &
            \href{https://www.bbva.com/es/mx/}{BBVA}\\ \hline
            
            {\bf C. D. Hernandez y J. R. López, "Burrobot", ISC. Trabajo Terminal 2019-A087, Escuela Superio de Cómputo, CDMX, México, 2019. } &

            Agente conversacional capaz de responder a preguntas frecuentes relacionadas a la gestión educativa por medio de la red social \textit{Facebook Messenger} integrado a la página interna de la Escuela Superior de Cómputo \textbf{ESCOM IPN MX}, brindado respuestas automatizadas dando solución en cortos periodos de tiempo con un servicio continuo y de bajo costo operacional &
            \begin{itemize}[leftmargin=*]
                \item Gran escalabilidad
                \item Tolerancia a fallos
                \item Bajo costo operacional
                \item Concurrencia de solicitudes
                \item Alta disponibilidad
                \item Prontitud de respuestas
                \item Robustez
            \end{itemize} &
            Prototipo para uso uso de la comunidad sin costo para el público pero con un costo estimado de \$240,556 &            \href{https://www.facebook.com/escomipnmx/}{ESCOM IPN MX}\\ \hline
        \end{longtable}

\newpage
\section{Descripción del documento}

    A lo largo de este trabajo terminal explicaremos a detalle cada uno de los capítulos que conforman este documento. A continuación daremos un resumen de cada uno de estos.

    \begin{description}
        \item[Capitulo 1 Introducción:] Resumen, planteamiento, objetivos, estado del arte y justificación del proyecto.
        \item[Capitulo 2 Marco conceptual:] Conceptos técnicos de los cuales se hará referencia en los siguientes capítulos, esto con la finalidad de tener un mayor entendimiento del tema.
        \item[Capitulo 3 Análisis del sistema:] Estudio de factibilidad, análisis de requerimientos, reglas del negocio, análisis de riesgos y finalmente los factores tomados en cuenta para la elección de las herramientas de software que emplearemos para el desarrollo del prototipo.
        \item[Capitulo 4 Diseño:] Diagramas de procesos, secuencia, casos de uso, clases, entidad-relación así como el diseño de la infraestructura con el que el servicio sera desplegado. 
    \end{description}



 
 